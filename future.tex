\mynewpage
\chapter{Research Directions}\label{sec:future:directions}

Our research is an effort of augmenting Multiple View Geometry to
handle general curved structures, including curves, surfaces and non-rigid phenomena.
The main lines of future research are: the theory and
practice of the multiview differential geometry of surfaces, and
the automatic curve-based calibration of multiple views. A more detailed list of
future work is given below.


\subsection{Theory}
\begin{itemize}
\item Curve-based calibration of 3 views. Research is underway in conjunction with prof.\ Peter Giblin to 
solve the problem of determining trinocular relative pose from
corresponding point-tangents across 3 views. Coupled with our single-view
pose calibration method from Section~\ref{sec:pose:from:curves}, this would allow for complete
curve-based structure from motion systems starting from a set of images 
without any initial calibration. Extending these ideas to include
curvature is another research possibility.
\item Futher study of nonrigid phenomena.
\end{itemize}

\subsection{Practice}
\begin{itemize}
\item Enrichment of the proposed 3D curve sketch with surface patches. 
The 3d curve sketch system is expected to form the initial building block in a broader effort to use
image evidence of the explicit geometry of curves and surfaces and reconstruct
these by integrating information across many views. The 3D curve sketch was
designed to be the initial scaffold on which surfaces may be constructed. It
forms a reliable structure from which to bootstrap a larger
reconstruction system that works under general conditions. One idea for
incorporating surfaces is the use of occlusion relationships to define a rough,
preliminary surface structure between 3D curves. This initial model can
then be optimized for photoconsistency to obtain a more precise surface model.
One way to carry this out is to use a hypothesize-and-test framework, where we
form hypotheses for whether there is a surface patch between two 3D curves, and
test these hypothesis by reprojecting onto views and verifying consistency.
Hypothesis formation could be image-based where we project the 3D curve sketch
onto an image (or a series of reference images), and form region fragments
between neighboring image curves. Each surface hypothesis could be tested by
forming a rough surface model between the underlying 3D curves of the patch
(\eg, a by fitting a minimal surface), and projecting the entire curve sketch
onto another view. Using the model for the surface patch, one can predict what
other curves should be occluded in case the hypothesis is correct. We then look
in the image for any edges that would support the curves that should have been
occluded. If there is support, then the curves are not actually occluded and the
surface hypothesis must be false.  The baisc idea of using image region
fragments has been successful as part of the top-performing 2-view stereo
algorithms in the literature.   
\item Image-based matching of curves in two and three views, so that these
correspondences can be used to bootstrap camera calibration from curves. 
We have done preliminary work using \sift\ descriptors attached to curves, and the
matching results were very promising. The descriptors are rotated to match the
tangent direction of the curve at each sample. There is room for improvement in terms of
efficiency, as we are computing \sift\ descriptors at different scales
for each curve sample, as well as matching all of them. Perhaps
a subsampling strategy should be used, or even a different strategy where the
histogram bins are placed on a global grid built around the entire curve.
\item Extension of the qualitative/geometric ideas of
Appendix~\ref{sec:epipolar} for epipolar geometry to curves. 
\item Development of a better numeric method to measure curvature derivative,
allowing to measure 3D torsion from image curves. Perhaps even a coarse but
robust measurement of curvature derivative could be useful in practice.
\end{itemize}

