\mynewpage
\chapter{Notation and Formulation}\label{sec:notation}
For the definitive kernel of this document, please refer to our 2016 paper at
the International Journal of Computer Vision~\cite{Fabbri:Kimia:IJCV2016}.
We have merged and extended that with more results from the literature,
as well as~\cite{Fabbri:PhD:2010}.

\section{Exotic Todo}
\begin{itemize}
\item Thinking about continous views vs discrete: are there names other than
  ``discretization'' to the link between the concept -- for instance, derivative
  -- and the real world, perhaps in category theory? Context - monads. Germs.
\item **** Frenet frame as differential variations instead of coordinates. Just
  reinterpretation of vectors. Differential operators.
\end{itemize}

\section{Modeling with geometric, symbolic and numeric notation}

\todo{This section will talk about my theory of planes of representation in
vector space geometry}

\todo{This will be a more advanced version, kept simultaneously with a more
didactic version as a lecture-style chapter ``What is a
matrix''~\cite{Fabbri:ALN}}

\subsection{Differential Geometry of Curves}
For our purposes, a 3D space curve $\Curve$ is a smooth map $S \mapsto
\Gama^w(S)$ of class $C^\infty$ from an interval
of $\mathbb{R}$
to $\mathbb{R}^3$, where $S$ is an arbitrary parameter,
$\tilde S$ is the arc-length parameter, and the superscript
$w$ denotes the so-called world coordinates (which can be any).\footnote{$\Gama^w$ is \emph{modulo}
Euclidean transformations $SE(3)$, \ie, the map $\rot\Gama^w + \transl$ gives
the same $\Curve$, for $\rot \in SO(3)$, and any 3-vector $\transl$,
\emph{if} $\rot$ and $\transl$ represent change of coordinate system and not
curve rotation -- this is up to the human modeler.}

The local Frenet frame of $\Curve$ in world coordinates is defined by the unit
vectors tangent $\T^w$, normal $\N^w$, binormal $\B^w$; $G$ is speed of
parametrization, curvature $K$, and torsion $\tau$. 

Similarly, a 2D curve $\gamma$ is a map $s \mapsto \gama(s)$ of class
$C^\infty$ from an interval of $\mathbb{R}$ to $\mathbb{R}^2$, where $s$ is an arbitrary parameter, $\tilde s$
is arc-length, $g$ is speed of parametrization,
$\t$ is (unit) tangent, $\n$ is (unit) normal, $\kappa$ is curvature, and $\kappa'$ is curvature derivative.
\footnote{The map $\gama$ (boldface) is also \emph{modulo} Euclidean
transformation SE(2), \emph{if} this transformation is meant to represent a change of
coordinate system, and not curve motion}

We will be concerned with regular curves, so that $G\neq 0$ and $g\neq 0$ unless
otherwise stated.
By classical differential geometry~\cite{Carmo:Diff:Geom}, we have
\begin{equation}\label{eq:TNB}
\left\{
\begin{aligned}
G  &= \|\Gama^{w'}\|  \\
\T^w &= \frac{\Gama^{w'}}{G}  &\hspace{3mm}
\N^w &= \frac{\T^{w'}}{\|\T^{w'}\|}  &\hspace{3mm}
\B^w &= \T^w \times \N^w \\
K  &= \frac{\|\T^{w'}\|}{G}  &
\dot{K} &= \frac{K'}{G}  &
\tau &= \frac{-\B^{w'}\cdot\N^w}{G},
\end{aligned}\right.
\end{equation} 
\begin{equation}
\begin{bmatrix}
\T^{w'}\\
\N^{w'}\\
\B^{w'}
\end{bmatrix} = 
G
\begin{bmatrix}
\hfill 0\,   & \hfill K &\,\,\, 0\ \  \\
-K & \hfill 0 \, &\,\,\, \tau\ \ \\
\hfill 0\, & -\tau\, &\,\,\, 0\ \ 
\end{bmatrix}
\begin{bmatrix}
\T^w\\
\N^w\\
\B^w
\end{bmatrix},\label{eq:frenet:explicit}
\end{equation}
and
\begin{equation}\label{eq:frenet2D:explicit}
\begin{aligned}
g  &= \|\gama'\|, &\hspace{7mm}
\t &= \frac{\gama'}{g}, &\hspace{7mm}
\n &= \t^\perp, &\hspace{7mm}
\kappa &= \frac{\t'\cdot \n}{g}, &\hspace{7mm}
\dot \kappa &= \frac{\kappa'}{g},
\end{aligned}
\end{equation} 
%\begin{equation}\label{eq:frenet2D:explicit}
%\hspace{-4.4in}\left\{
%\begin{aligned}
%g  &= \|\gama'\|\\
%\t &= \frac{\gama'}{g} & \hspace{3mm}
%\n &= \t^\perp\\
%\kappa &= \frac{\t'\cdot \n}{g} &
%\dot \kappa &= \frac{\kappa'}{g}
%\end{aligned}\right.
%\end{equation} 
where prime ``$\,'$'' denotes
differentiation with respect to an arbitrary spatial parameter ($S$ or $s$).
We use dot ``$\dot\ $'' 
to denote differentiation with respect to arc-length ($\tilde S$ or $\tilde s$)
only when an entity clearly belongs to either a space or an image
curve.
The matrix equations on the right of~\eqref{eq:frenet:explicit} are the Frenet equations. 
Note that both the curvature
derivatives $\dot{K}$ and $\dot \kappa$ are intrinsic quantities. For a slower
derivation of these entities, see Chapter~\ref{sec:curvemvg:extras}.

\subsection{Geometric Taylor Expansion$^*$ (skip on a first reading)}

\noindent The (geometric) Taylor expansion of $\Gama(s)$ for an arbitrary parameter $S$ is
\begin{align}\label{eq:taylor:s:grouped}
\Gama^w(S) = &\Gama^w_0 + S\,G_0\T_0^w + 
%
\frac{1}{2}S^2\left[G_0'\T_0^w + G_0^2K_0\N_0^w \right] +\\
%
&\frac{1}{6}S^3\left[ (G_0'' - G_0^3K_0^2)\T_0^w + (3G_0G_0'K_0 +
G_0^3\dot{K_0})\N_0^w +
G_0^3K_0\tau_0\B_0^w\right]+ \boldsymbol{O}(S^4),\notag
\end{align}
where the subscript 0 indicates evaluation at $S = 0$.
For the first order geometry, we have
\begin{align}
\T^w(S) &= \T_0^w + S\T_0^{w'} + \frac{S^2}{2}\T_0^{w''} + \boldsymbol{O}(S^3)\notag\\
      &= \T_0^w + S\,G_0K_0\N_0^w + \frac{S^2}{2}\, \left[(G_0'K_0 +
      G_0^2\dot{K_0})\N^w_0 -
      G^2_0 K_0^2\T_0^w +
      G^2_0K_0\tau_0\B^w_0\right] + \boldsymbol{O}(S^3) .\notag
\end{align}
Similarly, for second order geometry,
\begin{empheq}[left=\empheqlbrace]{align}
\N^w(S) &= \N_0^w + S\,G(-K\T^w + \tau\B^w) + \boldsymbol{O}(S^2)\\
K(S) &= K_0 + SG_0\dot{K}_0 + O(S^2)\\
\B^w(S) &= \T^w(S)\times\N^w(S)
\end{empheq}
and for third order geometry
\begin{align}
\tau(S) = \tau_0 + O(S).
\end{align}

\noindent All formulas in
Equation~\eqref{eq:frenet:explicit} would apply also to $\gama$ by
setting $\tau = 0$, except that the convention in 2D is to go counterclockwise to
define normals, as if there is a hidden $z$-axis pointing out of the 2D plane.

\subsection{Perspective Projection}
\begin{figure}
\centering
   \subfigure[]{ %
      \label{fig:1view}
      \includegraphics[height=2in]{figs/crv-mvg-1view.pdf}
    }
   \subfigure[]{ %
      \label{fig:mview:rig}
      \includegraphics[height=2.5in]{figs/multiview-rig.pdf}
    }
\caption{% 
The perspective projection of a space curve in (a) one view, and
(b) $n$ views.
}
\end{figure}
The projection of a 3D space curve $\Gamma$ into a 2D image curve $\gamma$ is
illustrated by Figure~\ref{fig:1view},
where the world coordinate system is centered at $O$ with basis
vectors $\{\e_1^w, \e_2^w, \e_3^w\}$. The
\emph{camera coordinate system} is centered at $\bc$
with basis vectors $\{\e_1, \e_2, \e_3\}$. A generic way of referring to
individual coordinates is by means of the specific subscripts $x,y$ and $z$ attached
to a symbol, \ie, $\boldsymbol{v} = [v_x, v_y, v_z]^\top$ for any vector
$\boldsymbol v$; other subscripts denote partial differentiation.

When describing coordinates in the
camera coordinate system we drop the $w$ superscript, \eg, $\Gama$ versus
$\Gama^w$, which are related by
\begin{equation}\label{eq:coord:transf:RT}
\Gama = \rot (\Gama^w - \bc) = \rot\Gama^w + \transl,
\end{equation}
where $\rot$ is a rotation and $\transl = -\rot\bc$
denotes the world coordinate origin in the camera coordinate system.
\nomenclature[04000]{$\transl(t)$}{
$(meters)$
The translation vector that is added when
passing from coordinates at time~$t$ to the world coordinates
}%

The projection of a 3D point $\Gama = [x,\, y,\, z]^\top$ onto the image plane
at $z=1$ is the point $\gama = [\uu,\,\vv,\,1]^\top$ related by
\begin{equation}\label{eq:projection} \Gama =
\depth\gama\,\,\,\,\text{or}\,\,\,\, [x,\, y,\, z]^\top =
[\depth\uu,\,\depth\vv,\,\depth]^\top,
\end{equation}
where we say that $\gama$ is in normalized image coordinates (focal distance is
normalized to 1), and the depth is $\depth = z = \e_3^\top\Gama$%
  \nomenclature[12000]{$\depth(\xi,\eta,t)$}{$(meters)$ Depth at pixel
  $(\xi,\eta)$ measured along $z$ axis of camera at time $t$: $\Gama =
  \depth\gama$}
  \nomenclature[00800]{$\uu,\vv$}{$(pixels)$ The horizontal and vertical coordinates of the image,
  respectively.}%
  %
  \nomenclature[10100]{$\gama(\xi,\eta,t)$}{
  (vector, $meters$) 
  The vector from the camera center to
  image point $(\xi,\eta)$ at time $t$, in the coordinates of the camera at time~$t$.
  }%
  %
  \nomenclature[10300]{$\gama^w(\xi,\eta,t)$}{
  (vector, $meters$) 
  $\gama(\xi,\eta,t)$ in world coordinates
  }%
%
from the third coordinate equation. Observe that image points are treated as 3D points
with $z = 1$. Thus, we can write
\begin{align}
\gama &= \frac{\Gama}{\depth}.
\label{eq:projection:isolated:gamma}
\end{align}
We note that $\f^\top\gama^{(i)} = 0$ and $\f^\top\Gama^{(i)} = \depth^{(i)}$,
where $\gama^{(i)}$ is the $i^{th}$ derivative of $\gama$ with respect to an
arbitrary parameter, for any positive integer $i$. Specifically, 
\begin{equation}\label{eq:depth:derivs}
\depth = z,\qquad \depth' = G T_z,\qquad \depth'' = G'T_z +
G^2K N_z.
\end{equation}
It is interesting to note that at near/far points of the curve, \ie, $\depth' =
0$, $T_z = 0$. \draftnote{vaguely reminds me of Lagrange multipliers}

In practice, normalized image coordinates $\gama = 
[\uu,\,\vv,\,1]^\top $ are described in terms of image
pixel coordinates $\gama_{im} = 
[x_{im},\, y_{im},\, 1]^\top$ through the 
intrinsic parameter matrix $\KK_{im}$ according to
\begin{equation}\label{eq:intrinsic:parameter:transf}
\gama_{im} = \KK_{im}\gama,
\ \ \ \ \ \
\KK_{im} = \begin{bmatrix}
\alpha_\uu & \sigma & \uu_o\\
0 &\alpha_\vv &  \vv_o\\
0 & 0 &  1
\end{bmatrix},
\end{equation}
where as usual $\uu_o$ and $\vv_o$ are the principal points, $\sigma$ is skew, and
$\alpha_\uu$ and $\alpha_\vv$ are given by the focal length divided by the width and
height of a pixel in world units, respectively.

%-------------------------------------------------------------

\subsection{Discrete and Continuous Sets of Views}
Two scenarios are considered. The first scenario consists of a
\emph{discrete set of views}
where a set of $n$ pinhole cameras observe a scene as shown in
Figure~\ref{fig:mview:rig}, with the last subscript in the symbols indentifying
the camera, \eg, $\gama_i$ denotes an image point in the $i^{th}$ camera,
and $\e_{3,i}$ denotes $\e_3$ in the $i^{th}$ view. The last index
may also be used as a superscript, $\eg$, $\gama_i^j$, which is 
consistent with Hartley and Zisserman's standard book~\cite{Hartley:Zisserman:multiple:view}.
The number of points in each image may, thus, be denoted as $n_p^j = n_{p,j}$,
where $j$ may be omitted if each image has the same number of points.
The number of images or cameras is denoted $n_c$.

The second scenario consists
of a \emph{continuous
set of views} from a continuously moving camera observing a space curve which may
itself be moving, $\Gama^w(S,t) = [x^w(S,t),\, y^w(S,t),\, z^w(S,t)]^\top$, 
where $S$ is the parameter along the curve and $t$ is time, 
described in the camera coordinate system associated with time $t$ as $\Gama(S,t) = \left[ x(S,t),\,
y(S,t),\, z(S,t) \right]^\top$, Figure~\ref{fig:mview:rig:continuous}. 
For simplicity, we often omit the parameters $S$ or $t$.
\begin{figure}
\centering
%  \scalebox{0.7}{\includegraphics{figs/multiview-rig-motion.eps}}
\scalebox{1.0}{\includegraphics{figs/multiview-rig-motion-inkscape.eps}}
\caption{% 
Multiview formulation of continuous camera motion and a possibly moving
contour.
}\label{fig:mview:rig:continuous}
\end{figure}
Let the camera position over time (\emph{camera orbit}) be described by the
space curve
$\bc(t)$ and the camera orientation
by a rotation matrix $\rot(t)$.
\nomenclature[02000]{$t$}{
$(seconds)$ Time, parametrizing camera or object motion
}%
\nomenclature[03900]{$\rot(t)$}{The rotation matrix from
coordinates at time~$t$ to the world coordinates}%
For simplicity, and without loss of
generality, we take the camera coordinate system at $t=0$ to be the world
coordinate system, \ie, 
$\bc(0) = 0$, $\transl(0) = 0$, and $\rot(0) = \id$,
where $\id$ is the identity matrix. Also, a stationary point can be modeled in
this notation by making $\Gama^w(t) = \Gama^w(0) = \Gama_0$.

A differential camera motion model using
time derivatives of $\rot(t)$ and $\transl(t)$ can be used to relate frames in a
small time interval. Since
$\rot\rot^\top = \id$,
\begin{equation}
\frac{d\rot}{dt}\rot^\top + \rot \frac{d\rot}{dt}^\top = 0,
\end{equation}
which implies that $\skewm \OO \doteq \frac{d\rot}{dt}\rot^\top$ is a
skew-symmetric matrix, explicitly written as
\begin{equation}
\skewm \OO = 
\begin{bmatrix}
0 & -\Omega_z & \Omega_y \\ \Omega_z & 0 & -\Omega_x \\ -\Omega_y & \Omega_x & 0 
\end{bmatrix},
\end{equation}
so that $\frac{d\rot}{dt} = \skewm\OO \rot$.
Denote $\OO = \left[\Omega_x,\, \Omega_y,\,
\Omega_z\right]^\top$ as a vector form characterization of $\skewm\OO$.
Similarly, the second-derivative of $\rot(t)$ is represented by only three
additional numbers $\frac{d\skewm\OO}{dt}$, so that
\begin{equation}\label{eq:rot:tt:atzero}
\frac{d^2\rot}{dt^2} = \frac{d\skewm \OO}{dt} \rot  + \skewm \OO
\frac{d\rot}{dt}  = 
%
\frac{d\skewm \OO}{dt} \rot  + \skewm \OO^2
\rot. 
\end{equation}
Thus, a second-order Taylor approximation of the camera rotation matrix using
$R(0) = \id$ is
\begin{equation}
\rot(t) \approx \id + \skewm \OO (0) t + \frac{1}{2} \left[ \frac{d\skewm \OO}{dt}(0) +
\skewm\OO^2(0) \right]t^2.
\end{equation}
%
Similarly, the camera translation can be described by a differential model
\nomenclature[04001]{$\OO = \frac{d\rot}{dt}(t)\rot^\top(t)$}{
($rad/s$)
The first-order approximation of
Rotation matrix, $\OO = \frac{d\rot(t)}{dt}$,  represented by rotation velocities
$[\Omega_x,\,\Omega_y,\,\Omega_z]^\top$ about the
x,y, and z axis, respectively.}%
%
\nomenclature[04500]{$\skewm \OO$}{(matrix, $rad/s$)
Entries of $\OO$ arranged into a
skew-symmetric matrix such that $\skewm \OO \mathbf v = \OO\times\mathbf
v$ for any vector $\mathbf v$\nomrefpage}%
%
\nomenclature[06000]{$\VV(t) = \frac{d\transl}{dt}(t)$}{(vector, $m/s$)
Tangential velocity vector to the curve
$\transl(t)$, \ie,~$\frac{d\transl}{dt}$}%
%
\nomenclature[07001]{$\VVspeed(t) = \norm{\frac{d\transl}{dt}(t)}$}{$(m/s)$ Tangential velocity to
the curve $\transl(t)$}%
%
\nomenclature[08000]{$\ttransl(t)$}{
(unit vector) Unit tangent vector to the curve $\bc(t)$, in camera coordinates
at time $t$.
}%
\nomenclature[08001]{$\ttransl^w(t)$}{
(unit vector) Unit tangent vector to the curve $\bc(t)$, in world coordinates.
}%
%
\begin{equation}\label{eq:vv:def}
\VV(t) \doteq \frac{d\transl}{dt}(t) = -\skewm\OO(t)\rot(t)\bc(t) -
\rot(t)\frac{d\bc}{dt}(t),
\qquad\qquad
\VV(0) = -\frac{d\bc}{dt}(0),
\end{equation}
and
\begin{equation}
\frac{d\VV}{dt}(t) = \frac{d^2\transl}{dt^2}(t) = -\frac{d^2 \rot}{dt^2}(t)\bc(t) -
2\frac{d\rot}{dt}(t)\frac{d\bc}{dt}(t) - \rot(t)\frac{d^2\bc}{dt^2}(t),
\end{equation}
which at $t=0$ gives $\frac{d\VV}{dt}(0) = -2\skewm\OO(0)\frac{d\bc}{dt}(0) -
\frac{d^2\bc}{dt^2}(0)$.

The choice of whether to adopt the Taylor approximation of $\bc(t)$ or
$\transl(t)$ as primary is entirely dependent in which domain the higher
derivatives are expected to diminish\footnote{Ric: this is to be debated
further}, giving
\begin{align}
\transl(t) \approx \VV(0)\,t + \frac{1}{2}\VV_{t}(0)\,t^2,
\qquad
\bc(t) \approx -\VV(0)t + \frac{1}{2} \left[ -\VV_t(0) +
2\skewm\OO(0)\VV(0) \right]t^2.
\end{align}

\begin{table}
  \renewcommand{\arraystretch}{1.4}
  \renewcommand{\tabcolsep}{0.1cm}
  \begin{center}
  \tiny
  \begin{tabular}{|c|l||c|l|}
  \hline
  \multicolumn{1}{|c|}{\textbf{Symbol}} &
  \multicolumn{1}{c||}{\textbf{Description}}&
  \multicolumn{1}{|c|}{\textbf{Symbol}} &
  \multicolumn{1}{c|}{\textbf{Description}}\\\hline\hline
  $\Gama^w$ & 3D point in the world coordinate system &
  $\t$ & Image curve tangent $\t = \gama'/g$\\\hline
  %
  $\Gama$ & 3D point in the camera coord. syst.  {\tiny$\Gama = \rot\Gama^w + \transl$} &
  $\n$ & Image curve normal $\n = \t^\perp$\\\hline
  %
  $\rot$ & Rotation matrix: world to camera coordinates &
  $\kappa$ & Curvature of the image curve $g\kappa\n = \t'$\\\hline
  %
  $\transl$ & Translation vector: world to camera coord. {\tiny $\transl =
  -R\bc$} &
  $S$, $\tilde S$ & Space curve arbitrary parameter \& arclength, resp.\\\hline
  %
  $\bc$ & The camera center &
  $G$ & Space curve speed of parametrization $G = \|\Gama'\|$\\\hline
  %
  $\skewm \OO$ & $\frac{d\rot}{dt} = \skewm \OO \rot$ &
  $\T$, $\T^w$ & Space curve tangent camera \& world coord., resp.\\\hline
  %
  $\OO$ & Vector form of the 3 entries of $\skewm \OO$ &
  $\N$, $\N^w$ & Space curve normal: camera \& world coord., resp.\\\hline
  %
  $\VV$ & $\VV = \frac{d\transl}{dt} = \skewm\OO\transl - \rot\bc_t$, also 
  $\VV = [V_x,\,V_y,\,V_z]^\top$
  &
  $\B$, $\B^w$ & Space curve binormal: camera \& world coord., resp.\\\hline
  %
  $\depth$ & Depth of image point $\Gama = \depth\gama$ &
  $\e_1$, $\e_2$, $\e_3$ & Basis vectors of the camera coordinate system\\\hline
  %
  $\gama$ & 2D point in normalized image coordinates &
  $\e_1^w$, $\e_2^w$, $\e_3^w$ & Basis vectors of the world coordinate system\\\hline
  %
  $\gama_{im}$ & 2D point in pixel image coordinates &
  $'$ & Diff. with resp. to $S$ or $s$, depending on context\\\hline
  %
  $s$, $\tilde s$ & Image curve arbitrary parameter \& arclength, resp. &
  $\dot\ $  & Diff. with resp. to arclength $\tilde S$ or $\tilde s$ \\\hline
  %
  $g$ & Image curve speed of parametrization $g = \|\gama'\|$ & 
  $\theta$ & The angle $\measuredangle(\T,\gama)$\\\hline
  $(u,v)$ & Image velocities $\gama_t = [u,\,v,\,0]^\top$ & &\\\hline
 \end{tabular}
 \end{center}\label{tab:notation} 
 \caption{Notation.} 
\end{table}


\subsection{Relating World and Camera-Centric Derivatives.} 

%\draftnote{Ben: I left this proposition in because it is used in some of the
%calculations later on. I eliminated the obvious part.}
%\begin{proposition}
%The velocity in camera coordinates $\vec v_t$ of a vector is related to its
%velocity in world coordinates $\vec v^w_t$ by
%\begin{equation}
%\vec v_t = \skewm \Omega\vec v + \rot\vec v^w_t,\qquad
%\vec v^w_t = \rot^\top(\vec v_t - \skewm \Omega\vec v).
%\end{equation}
%\end{proposition}
%\begin{proof}
%Taking the time derivative of $\vec v(t) = \rot(t)\vec v^w$ and using $\rot_t(t) =
%\skewm \Omega(t) \rot(t)$:
%\begin{align}
%\vec v_t &= \rot_t\vec v^w + \rot\vec v_t^w = \skewm\Omega \rot\vec v^w +
%\rot\vec v_t^w.
%\end{align}
%Substituting $\vec v^w = \rot^\top\vec v$, we get the result.
%\end{proof}

%\begin{corolary}
%The velocity of the camera center in world coordinates, $\bc_t$, is related to
%the velocity of the world coordinate system in camera coordinates $\VV = \transl_t$ by:
%\begin{equation}\label{eq:bc:from:VV:time}
%\VV = \skewm \Omega\transl - \rot\bc_t, \qquad
%\bc_t = \rot^\top(\skewm \Omega\transl - \VV).
%\end{equation}
%Hence, $\bc_t(0) = -\rot^\top\VV(0)$.
%\end{corolary}
%\begin{proof}
%Since $\transl(t) = -\rot\bc$ so that $-\transl(t) = \rot\bc$, we apply the previous theorem by making
%$\vec v = -\transl$ and $\vec v^w = \bc$.
%\end{proof}


%\begin{proposition}
%The velocity of a fixed point in space $\Gama^w(t) = \Gama_0$ in camera coordinates 
%$\Gama(t)$ at time $t$, is given by
%\begin{equation}
%\left\{\begin{aligned}
%\Gama_t &= \skewm\Omega \rot\Gama_0 + \VV & &\text{for any
%$t$},\\
%\Gama_t &= \skewm\Omega\Gama + \VV,& &\text{for $t=0$},
%\end{aligned}\right.
%\end{equation}
%where in the second formula all terms are evaluated at $t=0$.
%\end{proposition}
%\begin{proof}
%Differentiating~\eqref{eq:coord:transf:RT} with respect to $t$,
%\begin{equation}
%\Gama_t(t) = \frac{d\Gama(t)}{dt} = \frac{d\rot(t)}{dt}\Gama_0 +
%\frac{d\transl(t)}{dt}\nonumber
%= \skewm\Omega(t)\rot(t)\Gama_0 + \VV(t).
%\end{equation}
%\end{proof}

%
% TODO this draftnote might be useful in research:
%
%
%\draftnote{todo: do the same in~\eqref{eq:3D:point:velocity:allt} for $\Gama^w(t) \neq 0$;\\
%obs: When generalizing the formulas for any $t$, some authors~\cite{Ma:Soatto:etal:book} define $\mathcal V(t) :=  -
%\skewm\Omega(t)\transl(t)- \VV(t)$}%
%
%\noindent .
%It follows from Equation~\eqref{eq:3d:velocity:camera} that the first order
%approximation to $\Gama(t)$ at $t=0$ when the world coordinates are placed at
%the camera at $t=0$, so that $\rot(0) = \id$ and $\frac{d \rot}{dt} (0) = \skewm\Omega$, is
%\begin{equation}
%\Gama(t) \approx \Gama_0 + \left( \skewm\Omega(0)\Gama_0 - \VV(0) \right)t\,.
%\end{equation}

\subsection{Stationary and Non-Stationary Contours}\label{sec:apparent:contour:basics}
It is important to differentiate between image contours arising from a space
curve that is changing at most with a rigid transform (stationary contours),
\eg, reflectance contours and sharp ridges, 
and image curves arising from deforming space curves (non-stationary contours),
\eg, occluding contours,  the \emph{contour
generators} projecting to \emph{apparent contours}.
Stationary contours are characterized by $\Gama_t^w = 0$ while for 
occluding contours 
the viewing direction $\Gama(S,t)$ is tangent to the surface $\surface$ with
surface normal $\N$ ($\N^w = \rot^\top \N$)
\begin{align}\label{eq:occlusion:condition}
\Gama^\top\N = 0,\qquad \text{or}\qquad (\Gama^w-\bc)^\top\N^w = 0.
\end{align}
For the image curve $\gama(s,t)$ arising from 
the occluding contour, Figure~\ref{fig:mview:rig:continuous}, the
normal $\N$ to $\surface$ at an occluding contour~\cite{Giblin:Motion:Book} can
be consistently taken as $\N = \frac{\gama\times\t}{\|\gama\times\t\|}$.

Unless otherwise stated, we assume that the parametrization $\Gama^w(S,t)$ of $\surface$ is regular for
occluding contours, so that $\Gama^w_S(S,t)$ and $\Gama^w_t(S,t)$ form the
tangent plane to $\surface$ at $\Gama^w(S,t)$, and $t$ can be seen as a
spatial parameter~\cite{Giblin:Weiss:IVC1995}. The
correlation of the parametrization $S$ of $\Gamma$ at time $t$ to that of nearby
times is captured by
$\Gama_t^w(S,t)$, which 
is orthogonal to $\N^w$ (since $\N^w$ is orthogonal to the tangent plane),
but is otherwise arbitrary as a one dimensional choice. It is common to require
that
$\Gama^w_t(S,t)$ lay on the (infinitesimal) epipolar plane, spanned by
$\Gama^w(S,t)$, $\bc(t)$, and $\bc_t(t)$, referred to as the \emph{epipolar
parametrization}~\cite{Giblin:Motion:Book,Giblin:Weiss:IVC1995},
\begin{equation}\label{eq:epipolar:param:eq}
\Gama_t^w\times(\Gama^w-\bc) = 0,\qquad\text{or}\qquad \Gama_t^w =
\lambda(\Gama^w - \bc)\text{ for some $\lambda$.}
\end{equation}


\subsection{OLD BUT RELEVANT: Additional results relating world and camera-centric derivatives.} 

\begin{proposition}
Given any 3-\emph{vector} $v$, that may be time-dependent or not, then its camera
coordinates $\vec v$ are related to its world coordinates $\vec v^w$ by:
\begin{equation}
\vec v(t) = \rot(t)\vec v^w(t),
\end{equation}
and the velocity in camera coordinates $\vec v_t$ is related to the
velocity in world coordinates $\vec v^w_t$ by
\begin{equation}
\vec v_t = \skewm \Omega\vec v + \rot\vec v^w_t
\end{equation}
or, equivalently,
\begin{equation}
\vec v^w_t = \rot^\top(\vec v_t - \skewm \Omega\vec v)
\end{equation}
\end{proposition}
\begin{proof}
The vector is translation-independent, so that we can always find two points
such that $v = p_1 -p_2$ in 3D. Since we have
\begin{align}
\vec p_1 &= \rot\vec p^w_1 + \transl\\
\vec p_2 &= \rot\vec p^w_2 + \transl,\\
\intertext{then, subtracting, we get}
\vec p_1 -\vec p_2 &= \rot(\vec p^w_1-\vec p^w_2)\\
\vec v &= \rot\vec v^w.
\end{align}
The second part of the proof follows by taking the derivatives and using $\rot_t(t) =
\skewm \Omega(t) \rot(t)$:
\begin{align}
\vec v_t &= \rot_t\vec v^w + \rot\vec v_t^w\\
\vec v_t &= \skewm\Omega \rot\vec v^w + \rot\vec v_t^w\\
%
\shortintertext{substituting $\vec v^w = \rot^\top\vec v$,}
%
\vec v_t &= \skewm\Omega \vec v + \rot\vec v_t^w
\end{align}
\end{proof}
\begin{corolary}
The velocity of the camera center in world coordinates, $\bc_t$, is related to
$\VV = \transl_t$ by:
\begin{equation}\label{eq:VV:from:bc:time:deriv}
\VV = \skewm \Omega\transl - \rot\bc_t
\end{equation}
or, equivalently,
\begin{equation}
\bc_t = \rot^\top(\skewm \Omega\transl - \VV).
\end{equation}
Hence,
\begin{equation}
\bc_t(0) = -\rot^\top\VV(0).
\end{equation}
\end{corolary}
\begin{proof}
Since $\transl(t) = -\rot\bc$ so that $-\transl(t) = \rot\bc$, we apply the previous theorem by making
$\vec v = -\transl$ and $\vec v^w = \bc$.
\end{proof}



\begin{proposition}
The velocity of a fixed point in space $\Gama^w(t) = \Gama_0$ in camera coordinates 
$\Gama(t)$ at time $t$, is given by
\begin{align}
\Gama_t &= \skewm\Omega \rot\Gama_0 + \VV & &\text{for any $t$}\label{eq:3D:point:velocity},
%
\intertext{and}
%
\Gama_t &= \skewm\Omega\Gama + \VV,& &\text{for $t=0$}\label{eq:3D:point:velocity:at:zero},
\end{align}
where in the second formula all terms are evaluated at $t=0$.\footnote{Which is
why we drop the index in $\Gama_0$.}
\end{proposition}
\begin{proof}
Differentiating~\eqref{eq:coord:transf:t} with respect to $t$,
\begin{align}
\Gama_t(t) &= \frac{d\Gama(t)}{dt} = \frac{dR(t)}{dt}\Gama_0 +
\frac{d\transl(t)}{dt}\nonumber\\
%
&= \skewm\Omega(t)\rot(t)\Gama_0 + \VV(t).
\end{align}
The formula for $t=0$ is obtained by noting that $\rot(0) = \id$.
\end{proof}

\begin{proposition}
The velocity of a moving 3D point $\Gama^w(t)$, whose motion is parametrized by
$t$, in camera coordinates, is related to its velocity in world
coordinates by: 
\begin{equation}\label{eq:3D:point:velocity:allt}
\Gama_t = \skewm\Omega\Gama + \rot\Gama^w_t - \rot\bc_t(t).\ \ \ \ \text{for any $t$.}
\end{equation}
For a fixed point, we have that $\Gama^w_t = 0$, therefore its
relative velocity in camera coordinates is due solely to camera motion, given as:
\begin{equation}\label{eq:3d:velocity:camera}
\Gama_t = \skewm\Omega\Gama - \rot\bc_t.\ \ \ \text{for any $t$.}
\end{equation}
\draftnote{Ric: $\VV$ and $\VV_t$ as you have it do not have physical meaning;
$\bc_t$ and $\bc_{tt}$ do!}
\end{proposition}
\begin{proof}
Differentiating~\eqref{eq:coord:transf:t:anypoint} with respect to time,
\begin{align}
\Gama_t &= \rot_t\Gama^w + \rot\Gama^w_t + \transl_t\\
&= \skewm\Omega \rot\Gama^w + \rot\Gama^w_t + \VV
%
\intertext{But $\rot\Gama^w = \Gama - \transl$ from~\eqref{eq:coord:transf:t:anypoint}, so that}
%
\Gama_t &= \skewm\Omega(\Gama - \transl) + \rot\Gama^w_t + \VV\\
&= \skewm\Omega\Gama + \rot\Gama^w_t + \VV - \skewm\Omega\transl.
\end{align}
From~\eqref{eq:VV:from:bc:time:deriv} we have $\VV - \skewm\Omega\transl =
-\rot\bc_t$, and plugging this in gives the result.
\end{proof}
\draftnote{todo: do the same in~\eqref{eq:3D:point:velocity:allt} for $\Gama^w(t) \neq 0$;\\
obs: When generalizing the formulas for any $t$, some authors~\cite{Ma:Soatto:etal:book} define $\mathcal V(t) :=  -
\skewm\Omega(t)\transl(t)- \VV(t)$}%

\noindent If the world coordinates are placed at the camera at $t=0$, then
$\rot(0) = \id$ and $\frac{d \rot}{dt} (0) = \skewm\Omega$.
It follows from Equation~\eqref{eq:3D:point:velocity} that the first order
approximation to $\Gama(t)$ at $t=0$ is
\begin{equation}
\Gama(t) \approx \Gama_0 + \left( \skewm\Omega(0)\Gama_0 - \VV(0) \right)t\,.
\end{equation}

\subsection{OLD: Spherical Image Representation} The image point corresponding to $\Gama(t)$ in
image camera coordinates is denoted $\gama(t) =
\left[ \uu(t), \vv(t), 1 \right]^\top$, %
%
  \nomenclature[00800]{$\uu,\vv$}{$(pixels)$ The horizontal and vertical coordinates of the image,
  respectively.}%
  %
  \nomenclature[10100]{$\gama(\xi,\eta,t)$}{
  (vector, $meters$) 
  The vector from the camera center to
  image point $(\xi,\eta)$ at time $t$, in the coordinates of the camera at time~$t$.
  }%
  %
  \nomenclature[10300]{$\gama^w(\xi,\eta,t)$}{
  (vector, $meters$) 
  $\gama(\xi,\eta,t)$ in world coordinates
  }%
%
We can also represent an image point by the
corresponding point on the sphere centered on the camera focal point and with
the focal radius, \ie, with $\ugama = \frac{\gama}{\|\gama\|} =
\frac{\Gama}{\|\Gamma\|}$, and with
$\Gama = \udepth\ugama$. 

\begin{proposition}
Let $\gama = (\uu,\vv,1)^\top$ be an image point and denote $\ugama =
\frac{\gama}{\|\gama\|}$ the unit vector pointing to $\gama$ from the camera
focal point. Then a spherical representation of $\ugama$
\begin{equation}
\ugama = (\cos\phi\sin\theta,\,\sin\phi\sin\theta,\,\cos\theta)
\end{equation}
is related to $\gama$ by mapping $(\theta,\phi)$ to $(\uu,\vv)$ as follows:
\begin{empheq}[left=\empheqlbrace]{align}
\uu &= \cos\phi\tan\theta\\
\vv &= \sin\phi\tan\theta,
\end{empheq}
or mapping $(\uu,\vv)$ to $(\theta,\phi)$ using
\begin{empheq}[left=\empheqlbrace]{align}
\tan\phi &= \frac{\vv}{\uu}\\
\tan\theta &= \sqrt{\uu^2 + \vv^2},
\end{empheq}
or
\begin{empheq}[left=\empheqlbrace]{align}
(\cos\phi, \sin\phi) &= \frac{1}{\sqrt{\uu^2+\vv^2}} (\uu,\vv)\\
(\cos\theta, \sin\theta) &= \frac{1}{\sqrt{\uu^2+\vv^2 + 1}}
(1,\sqrt{\uu^2+\vv^2})\\
\end{empheq}
and 
\begin{equation}
\|\gama\| = \frac{1}{\cos\theta}.
\end{equation}
\end{proposition}
\draftnote{Ben: there is material on optical flow using the spherical
representation in Section~\ref{sec:optical:flow:survey}
.}
\begin{proof}
Equating
\begin{equation}
\frac{(\uu,\,\vv,\,1)}{\sqrt{\uu^2 + \vv^2 + 1}} =
(\cos\phi\sin\theta,\,\sin\phi\sin\theta,\,\cos\theta),
\end{equation}
we get
\begin{empheq}[left=\empheqlbrace]{align}
\frac{\uu}{\sqrt{\uu^2 + \vv^2 + 1}} &= \cos\phi\sin\theta\\
\frac{\vv}{\sqrt{\uu^2 + \vv^2 + 1}} &= \sin\phi\sin\theta\\
\frac{1}{\sqrt{\uu^2 + \vv^2 + 1}} &= \cos\theta,
\end{empheq}
and the results follow by deriving pairs of these equations.
\end{proof}

The vector $\Gama(t)$ can be represented in spherical coordinates using
the transformation
%-------- Slant/tilt/spherical angles
\begin{equation}
(x,y,z) = (\udepth\cos\phi\sin\theta,\udepth\sin\phi\sin\theta,\udepth\cos\theta)\,,
\end{equation}%
%
  \nomenclature[02900]{$\udepth(t),\,\theta(t),\,\phi(t)$}{($meters$ and
  $radians$) Polar coordinates in the camera coordinate system at time~$t$}%
  \nomenclature[11000]{$\ugama(\xi,\eta,t)$}{(unit vector) 
  $\gama(\xi,\eta,t)/\norm{\gama(\xi,\eta,\t)}$
  }%
  %
  \nomenclature[12100]{$\udepth(\xi,\eta,t)$}{(meters) 
  Depth at pixel $(\xi,\eta)$ measured along visual ray at time $t$: $\Gama =
  \udepth\ugama$}%
  %
  \nomenclature[11200]{$\hat\gama^w(\xi,\eta,t)$}{(unit vector)
  $\hat\gama(\xi,\eta,t)$ in world coordinates}%
%
which in inverse form is
\begin{empheq}[left=\empheqlbrace]{align}
\udepth(x,y,z) &= (x^2 + y^2 + z^2)^{1/2}\\
%
\theta(x,y,z) &= \tan^{-1}\left( \frac{\sqrt{x^2 + y^2}}{z} \right) =
  \tan^{-1}\left( \frac{\udepth}{z} \right)\\
%
\phi(x,y,z) &= \itan\left( \frac{y}{x} \right)\,,
\end{empheq}
see Figure~\ref{fig:spherical:coords}.

\subsection{OLD: Occluding Contours}\label{sec:apparent:contour:basics:old}
Occluding contours in 3D are also called \emph{contour generators}, while 
projected contour generators are called \emph{apparent contours}.
The only care we have to take when modeling occluding contours in our approach
is that the contour generator is not stationary.  We have to keep in mind the
difference between $\Gama^w(t)$ and $\Gama(t)$ in that 
$\Gama^w(t)\neq\Gama^w(0)$ and $\Gama^w(0)=\Gama(0)$.

The assignment of a
correspondence between a point $\Gama^w(s,t)$ and $\Gama^w(s,t+\Delta t)$ is at the
moment completely arbitrary. This assignment can be captured by the tangent
vector $\Gama^w_t(s,t)$, the limiting case of $\Gama^w(s,t+\Delta t) - \Gama^w(s,t)$.
Since this tangent vector is orthogonal to $\N$, the assingment of
correspondence is a one-dimensional choice, and it is common to require that
$\Gama^w_t(s,t)$ lies on the (infinitesimal) epipolar plane, which is spanned by
$\Gama(s,t)$, $\bc(t)$, and $\bc_t(t)$. This is called epipolar parametrization~\cite{Giblin:Motion:Book}.
Such epipolar condition would imply that the respective 3D occluding
points $\Gama^w(s,t)$ and $\Gama^w(s,\tDt)$ both lie in that epipolar plane,
even though they are not the same 3D point.  The tangent vector $\Gama_t^w$ lies in the tangent plane to the surface at the
point $\Gama^w$. If we also impose that $\Gama_t^w$ lies in the epipolar plane
spanned by $\Gama^w$, $\bc$, $\bc_t$, then it must be along the visual
direction.

\nomenclature[17000]{$n_c$}{Number of views, when considering discrete camera motion}
%
\nomenclature[17005]{$n_p^j = n_{p,j}$}{Number of points in view $j$, when considering discrete
sets of points}

\nomenclature[03550]{$\Gama_j$}{In the case of continuous camera motion, $\Gama_0$ is $\Gama$
at $t=0$ or simply $\Gama(0)$. In the case of discrete camera motion, $\Gama_j$
denotes the 3D point with index $j=1,\dots,n_p$.}% 
%
\nomenclature[10350]{$\gama^i_j$}{Coordinates of j-th point in view $i$}%
\nomenclature[10360]{$\gama^i$}{A 2D point in view $i$}%

\nomenclature[18100]{$el_k^i(p)$}{Epipolar line from point $p$ of view $k$ to view $i$}
\nomenclature[18110]{$eb_k^i(p)$}{Epipolar error band from point $p$ of view $k$ to view $i$}
\nomenclature[18120]{$e_k^i(p)$}{Epipole from view $k$ to view $i$}
\nomenclature[18000]{$F_{k,i}$}{Fundamental matrix from view $k$ to view $i$}

\mynewpage

%--------

%\begin{table}
%\hspace{-1.1in}
%\begin{tabular}{|c|c|l|} \hline
%\textbf{Symbol }   & \textbf{Units }       &\textbf{Description}\\
%     \hline\hline
%$\xi,\eta$ & pixels & The horizontal and vertical coordinates of the image,
%respectively \\\hline
%
%$t$ & seconds & Time, parametrizing camera or object motion\\\hline
%
%
%$\udepth^w,\,\theta^w,\,\phi^w$ & meters, radians & The polar coordinates of the world\\\hline
%
%$x(t),\,y(t),\,z(t)$ & meters & Coordinates in the camera coordinate system at time $t$\\\hline
%
%$\udepth(t),\,\theta(t),\,\phi(t)$ & meters & Polar coordinates in the camera coordinate system at time $t$\\\hline
%
%$\lbar\Gama = \left( \bar x,\,\bar y,\,\bar z \right)$ & meters (vector) & A
%point in the object coordinate system\\\hline
%
%$\Gama^w$ & meters (vector)& A point in world coordinates\\\hline
%
%$\Gama(t)$ & meters (vector)& A point in the coordinate system of the camera at
%time $t$\\\hline
%
%$R(t)$ & rotation matrix & The rotation matrix from coordinates at time $t$ to
%the world coordinates \\\hline
%
%$\mathcal{T}(t)$ & meters (vector)& 
%\begin{minipage}[l]{0.9\linewidth}
%The translation vector that is added when
%passing from coordinates at time $t$ to the world coordinates.
%
%
%\vspace{0.5mm}
%\end{minipage}
%\\\hline
%
%$\Omega = \frac{dR}{dt}(t)R^\top(t)$ & radian/second (vector)& 
%\begin{minipage}[l]{0.9\linewidth}
%The first-order approximation of
%Rotation matrix, $\Omega = \frac{dR(t)}{dt}$,  represented by rotation velocities
%$(\Omega_x,\,\Omega_y,\,\Omega_z)$ about the
%x,y, and z axis, respectively.
%\end{minipage}
%\\\hline
%
%$\skewm \Omega(t)$ & radian/second (matrix) & Entries of $\Omega$ arranged into a
%skew-symmetric matrix, $\skewm \Omega \mathbf v = \Omega\times\mathbf
%v$\\\hline
%
%$\VV(t) = \frac{d\transl}{dt}(t)$ & velocity vector & Tangential velocity vector to the curve
%
%$\transl(t)$, \emph{i.e.}, $\frac{d\transl}{dt}$\\\hline
%
%$\VVspeed(t) = \|\frac{d\transl}{dt}(t)\|$ & velocity & Tangential velocity to
%the curve $\transl(t)$\\\hline
%
%$\ttransl(t) = \frac{1}{\VVspeed}\frac{d\transl}{dt}(t)$ & unit vector & Unit tangent vector to the curve $\transl(t)$\\\hline
%
%$\bc(t) = -R^\top(t)\transl(t)$ & curve & 
%\begin{minipage}[l]{0.9\linewidth}
%The trajectory of the camera center in time $t$ expressed in
%world coordinates, also referred to as the camera orbit. 
%\end{minipage}
%\\\hline
%
%$I(\xi,\eta,t)$ & Intensity & Image intensity at a pixel $(\xi,\eta)$ at time $t$\\\hline
%
% & meters (vector) & 
%\begin{minipage}[l]{0.9\linewidth}
%The vector from the camera center to
%image point $(\xi,\eta)$ at time $t$, in the coordinates of the camera at time
%$t$.
%\vspace{1mm}
%\end{minipage}
%\\\hline
%
%$\ugama(\xi,\eta,t)$ & unit vector & $\gama(\xi,\eta,t)/\|\gama(\xi,\eta,\t)\|$\\\hline
%
%$\gama^w(\xi,\eta,t)$ & meters (vector) & $\gama(\xi,\eta,t)$ in world coordinates\\\hline
%
%$\hat\gama^w(\xi,\eta,t)$ & unit vector & $\hat\gama(\xi,\eta,t)$ in world coordinates\\\hline
%
%$\depth(\xi,\eta,t)$ & meters & Depth at pixel $(\xi,\eta)$ measured along $z$
%axis of camera at time $t$: $\Gama = \depth\gama$.\\\hline
%
%$\udepth(\xi,\eta,t)$ & meters & Depth at pixel $(\xi,\eta)$ measured along
%visual ray at time $t$: $\Gama = \udepth\ugama$.\\\hline
%
%$\field(\xi,\eta,t)$ & velocity vector $(m/s)$ & The image velocity vector of a
%point due to object-camera relative motion\\\hline
%
%$\ufield(\ugama,t)$ & 
%velocity vector $(rad/s)$ & 
%\begin{minipage}[l]{0.9\linewidth}
%\vspace{0.5mm}
%The velocity vector of a
%point's visual direction due to object-camera relative motion
%\vspace{0.5mm}
%\end{minipage}
%\\\hline
%
%$u,\,v$ & $m/s$ & 
%\begin{minipage}[l]{0.9\linewidth}
%$\field = (u,\,v)$, obtained from $\frac{d\gama}{dt} = (u,v,0)^\top$, the extrinsic representation of the
%velocity field in image coordinates.
%\vspace{0.5mm}
%\end{minipage}
%\\\hline
%
%$\hat u,\,\hat v$ & $rad/s$ & 
%\begin{minipage}[l]{0.9\linewidth}
%$\ufield = (\hat u,\,\hat v)$, obtained from
%$\frac{d\ugama}{dt}(\theta(t),\phi(t))$, the extrinsic representation of the
%velocity field in spherical image coordinates.
%\vspace{0.5mm}
%\end{minipage}
%\\\hline
%
%$\N$ & unit vector & Normal to the surface of the object\\\hline
%
%$\sigma$ & radians & Slant -- $\measuredangle(\N,\ugama)$, the angle between the normal $\N$ and the visual
%direction $\ugama$\\\hline
%
%$\nu$ & radians & 
%\begin{minipage}[l]{0.9\linewidth}
%\vspace{0.5mm}
%Tilt -- $\measuredangle \left(\left[\N -
%(\N\cdot\ugama)\N\right],\,x\text{-axis}\right)$, the angle between the projected normal and the $x$-axis.
%\vspace{0.5mm}
%\end{minipage}
%\\\hline
%
%$\kappa_1,\,\kappa_2$ & $1/m^2$ & Principal curvatures of the
%surface\\\hline
%
%$\mathbf L$ & meters (vector) & Position of light source in the world coordinates\\\hline
%
%$\albedo$ & unit-less & Reflectance (albedo)\\\hline
%$e_1,\,e_2,\,e_3$ & unit vectors & 
%\begin{minipage}[l]{0.9\linewidth}
%\vspace{0.5mm}
%Canonical basis in any coordinate system,
%$e_1 = (1,\,0,\,0)^\top$, $e_2 = (0,\,1,\,0)^\top$, $e_2 = (0,\,0,\,1)^\top$
%\vspace{0.5mm}
%\end{minipage}
%\\\hline
%\end{tabular}
%\caption{Table of Notation.}
%\end{table}

\subsection{Multiview Differential Geometry of Surfaces: Notation}\label{sec:dg:surf:notation}

\indraftnote{
todo
\begin{itemize}
\item Include the figures about the numerical experiments with pov-ray
\end{itemize}
}
In this manuscript, up to this date, we present a preliminary study and theoretical results on
the local differential geometry of surfaces in multiple views, including
surface-induced image flow and the multiview behavior of shading.
Chapter~\ref{sec:surf:geometry} reviews useful background material related to
the differential geometry of surfaces in the context of 3D vision.

The object local coordinate system is defined to be centered at a select point on the
surface of the object, $\Gama_c(0) =
\depth_{c}(0)\left[\uu_c(0),\vv_c(0),1\right]^\top$, and with the direction
$\bar z$ aligned with the surface normal and with $\bar x$ and $\bar y$ defined
as the direction of principal curvatures $\kappa_1$ and $\kappa_2$,%
%
  \nomenclature[14000]{$\kappa_1,\,\kappa_2$}{($1/m^2$) Principal curvatures of
  the surface}
%
assuming they are different from each
other (non-umbilical points), and such that the resulting coordinate system has
same orientation as the camera coordinate system. 

A point in local coordinates
is hence denoted
$\lbar \Gama = \left[\bar x, \bar y, \bar z\right]^\top$.  %
\nomenclature[02950]{$\lbar\Gama = \left( \bar x,\,\bar y,\,\bar z \right)$}{
$(meters)$ A point in the object coordinate system}
The object surface near the origin $\lbar \Gama_c = \left[0,0,0\right]^\top$ 
is then described in Monge patch form as:
\begin{equation}\label{eq:BB:paraboloid:taylor}
\bar z = \frac{1}{2}\kappa_1\bar x^2 + \frac{1}{2}\kappa_2\bar y^2 + \cdots
\end{equation}
We define the curvature matrix $\mathbf K$ as:\begin{equation}
\mathbf K = \begin{bmatrix}\label{eq:BB:K:matrix}
-\kappa_1 & 0 & 0\\
0 & -\kappa_2 & 0\\
0 & 0 & 0
\end{bmatrix},
\end{equation} 
which is the canonical matrix for a paraboloid, and
write~\eqref{eq:BB:paraboloid:taylor} as
\begin{equation}\label{eq:BB:quadratic:matricial:taylor}
e_3^\top
\overline \Gama
+
\frac{1}{2}
\overline \Gama^\top 
\mathbf K
\overline \Gama
+\dots = 0
\end{equation}
where we use the notation
%
\begin{align}
e_1  = 
\begin{bmatrix}
1\\ 0\\ 0 
\end{bmatrix}
\,\,\,
e_2  = 
\begin{bmatrix}
0\\ 1\\ 0 
\end{bmatrix}
\,\,
e_3  = 
\begin{bmatrix}
0\\ 0\\ 1 
\end{bmatrix}.
\end{align}%
  \nomenclature[16000]{$\e_1,\,\e_2,\,\e_3$}{(unit vectors)
  Canonical basis in any coordinate system,
  $\e_1 = (1,\,0,\,0)^\top$, $\e_2 = (0,\,1,\,0)^\top$, $\e_2 = (0,\,0,\,1)^\top$
  }%
%
%
Later in this chapter we assume the object can locally be approximated by
truncating the above equation to second-order, obtaining a special quadric
surface model $\surf$.  This second-order surface approximation is an elliptic
paraboloid if $\kappa_1 \kappa_2 >
0$, a hyperbolic paraboloid if $\kappa_1 \kappa_2 < 0$, or a plane if $\kappa_1
= \kappa_2 = 0$.
The 3D points written in the camera coordinates at time~$t$ can be written in
terms of the Monge patch coordinates as in Figure~\ref{fig:project-quadric},
\begin{figure}
\centering
\includegraphics[width=4.5in]{figs/quadric-rig.eps}
\caption{Projection of a quadric in images.}
\label{fig:project-quadric}
\end{figure}
%
\begin{equation}
%
\Gama(t)
=
\overline{R}(t)\overline\Gama(t)+\overline{\transl}(t)
\label{eq:BB:worl:to:camera}
%
\end{equation}
%
%
where
%
\begin{align}
\overline{\rot}(t)&=\rot(t)\overline R_0 \\
\overline{\transl}(t)&=\rot(t)\overline \transl_0 + \transl(t)\label{eq:BB:tbart:from:tbar0}
\end{align}
%
We also have the following notation:
\begin{align}
\overline \rot_0 = 
\begin{bmatrix}
a_1 & a_2 & a_3\\
b_1 & b_2 & b_3\\
c_1 & c_2 & c_3
\end{bmatrix}\,\,\,\,\,\,
\overline \transl(t) = 
\begin{bmatrix}
\overline \transl_x\\
\overline \transl_y\\
\overline \transl_z
\end{bmatrix}.
\end{align}

\noindent The unit normal $\N = (\N_x,\,\N_y,\,\N_z)$%
  \nomenclature[14000]{$\N$}{(unit vector)
  Normal to the surface of the object, usually written in camera coordinates
  unless otherwise stated.
  }
%
to the surface, written in camera coordinates, is determined by the
slant and tilt angles (more later). The world coordinates for the normal will be
denoted $\N^w$, and we can write $\N = R\N^w$.
  \nomenclature[14005]{$\N$}{(unit vector)
  Normal to the surface of the object, usually written in camera coordinates
  unless otherwise stated.
  }
\draftnote{TODO ASAP: parametrize normal in terms of tilt and slant}

The third column of the rotation matrix $\overline R(t)$ is the normal $\N$ written
in camera coordinates at time~$t$:
\begin{equation}
\N(t) = \overline R(t)\e_3
\end{equation}
where $\e_3$ is the normal in local coordinates of the surface patch. Therefore,
only two angles, \emph{e.g.}\ slant $\sigma$%
%
  \nomenclature[13100]{$\sigma$}{$(radians)$ Slant --
  $\measuredangle(\N,\ugama)$, the angle between the normal $\N$ and the visual
  direction $\ugama$}
%
and tilt $\nu$%
%
  \nomenclature[13150]{$\nu$}{$(radians)$ Tilt -- $\measuredangle \left(\left[\N
  - (\N\cdot\ugama)\N\right],\,x\text{-axis}\right)$, the angle between the
  projected normal and the $x$-axis.
  }
%
, appear in the third
column. The first two columns are the camera coordinates of unit vectors along the
$\bar x$, and $\bar y$ directions, and this is completely defined by the normal
and an in-plane rotation $\psi$. This is given in more details as follows.
The surface normal may be described in terms of slant
$\sigma$ and tilt $\nu$, 
\begin{equation}\label{eq:BB:N:slant:tilt}
\N = \left( \cos\nu\sin\sigma,\,\sin\nu\sin\sigma,\cos\sigma \right)^\top,
\end{equation}
see Figure~\ref{fig:slant:tilt}. 

\begin{figure}
\centering
   \subfigure[]{ %
    \label{fig:spherical:coords}
      \begin{minipage}[c]{0.46\linewidth}%
         \centering
         \scalebox{0.33}{\includegraphics{figs/Spherical_coordinates-norhobar.eps}}
      \end{minipage}}
    \subfigure[]{ %
      \label{fig:slant:tilt}
      \begin{minipage}[c]{0.46\linewidth}%
         \centering
         \scalebox{0.33}{\includegraphics{figs/slant-tilt.eps}}
      \end{minipage}}
\caption{% 
(a) Spherical coordinates used in this text, and (b) The
orientation of a (tangent) plane with respect to the camera plane can be given
by the \emph{slant} -- which is the angle between the normal of the plane and
the viewing direction -- and the tilt -- which is the angle the projected normal
makes with the $x$-axis of the camera coordinate system (from~\cite{Forsyth:Ponce:Book}).
}
\end{figure}

We define $\psi$ as the in-plane rotation of the
local $x$ and $y$ axes with respect to a predefined reference coordinate frame of the
tangent plane. We now derive an explicit expression for this reference
coordinate frame in terms of the camera frame, with the intention of deriving a
parametrization of the rotation matrix by the angles $\sigma,\nu,\psi$.
The reference frame for measuring $\psi$, denoted $\e_1^\N,\e_2^\N$, is conveniently 
defined with respect to the camera frame: $\e_1^\N$ is the orthogonal projection of $\e_1$ onto
the tangent plane with normal $\N$, and $\e_2^\N$ is choosen to form a
right-handed frame $(\e_1^\N,\e_2^\N,\N)$. This can be written analytically as:
\begin{equation}
\left\{
\begin{aligned}~\label{eq:BB:ref:vecs:e:N} 
\e_2^\N &\doteq \N\times\e_1\\
\e_1^\N &\doteq \e_2^\N\times\N,
\end{aligned}
\right.
\end{equation}
which allows us to express the local $x$ and $y$ direction vectors $\bar \e_1$
and $\bar \e_2$, respectively, as:
\begin{equation}~\label{eq:BB:local:vecs:ref}
\left\{
\begin{aligned}
\bar \e_1 &= \cos\psi\,\e_1^\N + \sin\psi\,\e_2^\N + 0\,\N\\
\bar \e_2 &= \sin\psi\,\e_1^\N - \cos\psi\,\e_2^\N + 0\,\N.
\end{aligned}
\right.
\end{equation}
Using~\eqref{eq:BB:N:slant:tilt} into~\eqref{eq:BB:ref:vecs:e:N}, we can write the
reference vectors in terms of $\sigma$ and $\nu$
\begin{equation}\label{eq:BB:eq:ref:vecs:slant:tilt:psi}
\left\{
\begin{aligned}
\e_2^\N &= \left( 0,\,\cos\sigma,-\sin\nu\sin\sigma \right)^\top\\
\e_1^\N &= \left( -\sin^2\nu\sin^2\sigma - \cos^2\sigma,\,\sin\nu\sin\sigma,-\cos\sigma \right)^\top,
\end{aligned}
\right.
\end{equation}
which we plug into~\eqref{eq:BB:local:vecs:ref} to yield the principal directions in
terms of $\sigma$, $\nu$, and $\psi$:
\begin{equation}\label{eq:BB:eq:principal:direcs:slant:tilt:psi}
\bar \e_1 = 
\begin{bmatrix}
-\cos\psi\sin^2\nu\sin^2\sigma - \cos\psi\cos^2\sigma\\
\sin\psi\sin\nu\sin\sigma - \cos\psi\cos\sigma\\
-\cos\psi\cos\sigma-\sin\psi\sin\nu\sin\sigma 
\end{bmatrix},\ 
\bar \e_2 = 
\begin{bmatrix}
-\sin\psi\sin^2\nu\sin^2\sigma - \sin\psi\cos^2\sigma\\
\sin\psi\sin\nu\sin\sigma - \cos\psi\cos\sigma\\
-\sin\psi\cos\sigma + \cos\psi\sin\nu\sin\sigma
\end{bmatrix},
\end{equation}
written in camera coordinates.  Then, a point $\Gama(0)$ is related to $\lbar \Gama$ as
\begin{equation}
\Gama(0) = \lbar R_0\lbar \Gama + \lbar T_0,
\end{equation}
where
\begin{align}
\overline R_0 = 
\begin{bmatrix}
-\cos\psi\sin^2\nu\sin^2\sigma - \cos\psi\cos^2\sigma
&-\sin\psi\sin^2\nu\sin^2\sigma - \sin\psi\cos^2\sigma & \cos\nu\sin\sigma\\
\sin\psi\sin\nu\sin\sigma - \cos\psi\cos\sigma        &\sin\psi\sin\nu\sin\sigma - \cos\psi\cos\sigma        & \sin\nu\sin\sigma\\
-\cos\psi\cos\sigma-\sin\psi\sin\nu\sin\sigma         &-\sin\psi\cos\sigma + \cos\psi\sin\nu\sin\sigma       & \cos\sigma
\end{bmatrix},
\end{align}
and $\lbar\transl_0 = \left( \lbar\transl_x,\,\lbar\transl_y,\,\lbar\transl_z \right)^\top$.
Note that  the first two columns of the camera matrix are the principal
direction vectors written in camera coordinates and the third column is the
normal in camera coordinates, all parametrized by $\sigma$, $\nu$ and $\psi$.

\begin{proposition}
Given a paraboloid 
$\frac{1}{2}\lbar{\Gama}^\top\mathbf K\lbar{\Gama}
+ \e_3^\top\lbar{\Gama} = 0$ in the coordinate system defined by $\Gama_0 = \lbar
R_0\lbar{\Gama}_0 + \lbar\transl$,
and a camera motion defined by
$R(t)$ and $\transl(t)$. Then, the equation of the paraboloid in camera
coordinates is given by:
\begin{equation}\label{eq:BB:paraboloid:xyz:thru:gama}
\Gama^\top \mathbf K_c \Gama -2\overline \transl^\top \mathbf K_c\Gama + e_3^\top \overline R^\top\Gama +
 \overline \transl^\top \mathbf K_c \overline \transl - e_3^\top \overline
 R^\top \overline \transl = 0\,.
\end{equation}
where $\mathbf K_c = \frac{1}{2}\overline R \mathbf K \overline R^\top$, 
$\mathbf K$ is the principal curvature matrix of the surface as defined in
Equation~\eqref{eq:K:matrix}, $\overline R = R(t)\overline R_0$, and $\overline
\transl = R(t)\overline\transl_0 + \transl(t)$.
\end{proposition}
\begin{proof}
The main equations are:
\begin{empheq}[left=\empheqlbrace]{align}
\label{eq:BB:quadratic:matricial:b}
\frac{1}{2}\overline \Gama^\top 
&
\mathbf K
\overline \Gama
+ 
e_3^\top
\overline \Gama = 0
&
&\text{\small{(second-order surface model)}}
\\
%
\Gama_0 &= \overline R_0\overline\Gama + \overline \transl_0 &
&\text{\small{(Monge to camera coordinates at $t=0$)}}
\label{eq:BB:gamat:from:gamabar}
\\
%
\label{eq:BB:gamat:from:gamazero}
\Gama(t) &= R(t)\Gama_0 + \transl(t)
& &\text{\small{(camera coordinates at time t)}}\\
%
\Gama(t) &= \depth(t)\gama(t)& &\text{\small{(projection equation)}}\label{eq:BB:projection:B}
\end{empheq}
%
%
We have from~\eqref{eq:BB:gamat:from:gamabar} and~\eqref{eq:BB:gamat:from:gamazero}
\begin{align}
\Gama(t) &= R(t)\overline R_0\overline\Gama + R(t)\overline\transl_0 +
  \transl(t)\nonumber\\
&= \lbar R\lbar\Gama + \lbar\transl\,,
\end{align}
%
giving
%
\begin{equation}
\overline \Gama = \overline R(t)^\top \Gama(t) - \overline R(t)^\top \overline
\transl(t)\,.
\end{equation}
Substituting this into the above equation~\eqref{eq:BB:quadratic:matricial:b}
(for simplicity, the dependency on $t$ will not be written), we have
%
\begin{align}
\frac{1}{2}\left[ \overline R^\top\Gama - \overline R^\top\overline \transl \right]^\top 
\mathbf K
\left[ \overline R^\top\Gama - \overline R^\top\overline \transl \right]
%
 + e_3^\top
\left[ \overline R^\top\Gama - \overline R^\top\overline \transl \right]
=0\,,
%
\end{align}
or
\begin{align}\label{eq:BB:paraboloid:xyz:trhu:gama:concise}
\frac{1}{2}\left[ \Gama - \overline \transl \right]^\top \overline R \mathbf K  \overline R^\top \left[ \Gama -
\overline \transl \right] 
%
+ e_3^\top \overline R^\top \left[\Gama - \overline \transl\right] = 0\,.
\end{align}
Using $\mathbf K_c := \frac{1}{2}\overline R \mathbf K \overline R^\top$,
distributing, and rearranging, we get the final equation.
\end{proof}



We are interested in an image neighborhood around a given center point
$\gama_c(0) = \left[ \uu_c(0), \vv_c(0), 1 \right]^\top$ at $t=0$. The paraboloid approximation would
then be around the 3D point $\Gama_c(0) = \left[ x_c(0), y_c(0), z_c(0) \right]^\top$,
with the notation $\Gama_c(t) := \depth_c(t)\gama_c(t)$. In local coordinates, $\Gama_c(0)$ corresponds to 
$\overline \Gama = \left[ 0, 0, 0 \right]^\top$, which, when plugged
into~\eqref{eq:BB:worl:to:camera}, yields:
\begin{equation}
\Gama_c(0) = \overline \transl(0) = \overline\transl_0
\end{equation}
It is then clear that $\overline \transl(t)$ is the coordinates of the origin of the Monge
patch with respect to the camera coordinates at time~$t$.
From~\eqref{eq:BB:tbart:from:tbar0}, we can write:
\begin{align}
\overline{\rot}(t)&=\rot(t)\overline \Gama_c(0) + \transl(t)\\
\overline{\transl}(t)&=\rot(t)\depth_c(0)\gama_c(0) + \transl(t)
\end{align}
Note that $\overline \transl(t) \neq \Gama_c(t)$, because $\Gama_c$ was defined with
respect to the center of the image patch, $\gama_c$, which for any $t \neq 0$
will not correspond to the origin of the 3D monge patch.
Thus, given the image point $\gama_c(0)$ at $t=0$, the translation to the Monge patch is
determined from a single number $\depth_c(0)$, so in total we need 6 parameters to
determine our local surface model: 
\begin{equation}
\mathcal M = \left\{ \depth_c(0), \sigma,\nu,\psi, k_1, k_2 \right\}
\end{equation}
